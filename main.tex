\documentclass[12pt,a4paper]{article} 
% Draft debuguje ale psuje obrazki
%\showboxbreadth=\maxdimen % debug
%\showboxdepth=\maxdimen % debug
%\usepackage{lua-visual-debug} % debug

\usepackage[a4paper,twoside,top=2.5cm,bottom=2cm,inner=3.5cm,outer=2cm]{geometry}
\usepackage{setspace}
\usepackage[hidelinks]{hyperref}

\usepackage{fontspec} % Dla Arial na stronie tytułowej
\newfontfamily\titlepagefont{Arial} % Ustalenie czcionki dla strony tytułowej
%\setstretch{1.5} 

\usepackage{newtxtext} % Czcionka New Times Roman
\usepackage{graphicx}
\usepackage{adjustbox}
\usepackage[utf8]{inputenc}
\usepackage[english, polish]{babel}
\babelprovide[transforms = oneletter.nobreak]{polish} 
\usepackage{array}
\usepackage{enumitem}
\usepackage[backend=biber, defernumbers=true]{biblatex}
\usepackage{tabularx}
\usepackage{xltabular}
\usepackage{multirow}
\usepackage{stackengine}
\usepackage{tocloft}
\usepackage{enumitem}
\usepackage{chngcntr}
\usepackage{newfloat}
\usepackage{float}
\usepackage[font=normalsize]{caption}
\usepackage[justification=centering]{caption}
\usepackage{boldline}
\usepackage{indentfirst}
\usepackage{listings}
\usepackage{csquotes}
\usepackage{pdflscape}


% komentarze
\usepackage[T1]{fontenc}
\usepackage{lmodern}
\usepackage{xcolor}
\usepackage[breakable]{tcolorbox}
\tcbuselibrary{breakable}
\setlength{\marginparwidth}{2cm} % Fix todonotes margin warning
\usepackage{todonotes}

% justified
% \usepackage[final]{microtype}

\tcbset{
  colframe=red, % Color of the frame
  colback=white, % Background color
  boxrule=0.5mm, % Frame thickness
  arc=2mm, % Rounded corners
  boxsep=2pt, % Padding inside the box
  left=2pt, % Left padding
  right=2pt, % Right padding
  top=2pt, % Top padding
  bottom=2pt % Bottom padding
}

\newtcolorbox{komentarz}{
  colframe=red, % Red frame
  colback=white, % White background
  boxrule=0.5mm, % Frame thickness
  arc=2mm, % Rounded corners
  boxsep=2pt, % Internal padding
  left=2pt, right=2pt, top=2pt, bottom=2pt, % Padding adjustments
  width=\linewidth, % Full width of text area
  breakable % Allow breaking across pages
}



\DeclareFloatingEnvironment{graph}

\addbibresource{bib/thesis.bib}

\newcolumntype{C}[1]{>{\centering\arraybackslash} m{#1} }
\newcolumntype{?}{!{\vrule width 1.5pt}}

\addto\captionsenglish{
  \renewcommand{\figurename}{Rysunek}
  \renewcommand{\thefigure}{\arabic{section}.\arabic{figure}}
  \renewcommand{\graphname}{Listing}
  \renewcommand{\thegraph}{\arabic{section}.\arabic{graph}}
  \renewcommand{\contentsname}{Spis treści}
}


\addto\captionsenglish{
    \renewcommand{\figurename}{Rysunek}
    \renewcommand{\graphname}{Listing}
    \renewcommand{\thegraph}{\arabic{section}.\arabic{graph}}
    \renewcommand{\contentsname}{Spis treści}
}


\renewcommand{\thefigure}{\arabic{section}.\arabic{figure}}

\newcommand{\row}[1]{\hline #1 \\}
\newcommand{\image}[2]{\begin{figure}[h!]
    \centering
    \includegraphics[width=\linewidth]{#1}
    \caption{#2}
    \label{fig:image}
\end{figure}
\par}

\newcommand{\img}[3]{\begin{figure}[#3]
    \centering
    \includegraphics[width=\linewidth]{#1}
    \caption{#2}
    \label{fig:image}
\end{figure}
\par}


\newcommand{\lis}[3]{\begin{graph}[h!]
    \centering
    \caption{#2}
    \includegraphics[#3]{#1}
    \label{fig:graph}
\end{graph}
\par}

\def\blankpage{%
      \clearpage%
      \thispagestyle{empty}%
      \null%
      \clearpage}

\lstset{
    basicstyle=\small\ttfamily, % Adjust the font size here
   % framerule=0.5pt,
}

\newcommand{\newCharapter}[1]{\newpage \include{#1}}


\begin{document}
\pagenumbering{gobble} % no page numbers before TOC


\counterwithin{figure}{section}
\counterwithin{lstlisting}{section}
\counterwithin{table}{section}

\begin{titlepage}
    \titlepagefont
    \hspace*{-0.5cm}{\includegraphics[width=7cm]{img/title/weii-2.jpg}} \\
    \vspace{4cm}
    \break
    \fontsize{1.4cm}{1.4cm}\titlepagefont {Praca dyplomowa} \\
    \fontsize{1.4cm}{1.4cm}\titlepagefont {inżynierska} \\
    \vspace{0.5cm}
    \break
    \normalsize{} na kierunku Informatyka \\
    na specjalności Inżynieria oprogramowania \\
    \vspace{1cm}
    \break
    \large
    Aplikacja internetowa domu aukcyjnego z interaktywną wizualizacją obiektów 3D \\
    \break
    Auction house web application with interactive 3D object visualisation \\
    \vspace{1.5cm}
    \break
    Konrad Tomasz Nowak \\
    99640   \\[0.3cm]
    Maciej Ołdakowski \\
    99648   \\[0.3cm]
    Patryk Kamil Nowacki \\
    99638   \\[0.3cm]
    \vspace{1.5cm}
    \break
    \normalsize{}
    Promotor Dr Marcin Barszcz
    \vspace*{\fill}
    \begin{figure*}[bbp]
        {\titlepagefont Lublin 2025}
    \end{figure*}
\end{titlepage}

\onehalfspacing
\blankpage

\newCharapter{chapters/0abstract}
\setcounter{figure}{0}

\clearpage
\pagenumbering{arabic} % start numbering
\setcounter{page}{1}   % make TOC page number 1
\tableofcontents
\newCharapter{chapters/1wstep}
\setcounter{figure}{0}
\newCharapter{chapters/2cel_zakres_pracy}
\setcounter{figure}{0}
\newCharapter{chapters/3analiza_rynku}
\setcounter{figure}{0}
\newCharapter{chapters/4technologie}
\setcounter{figure}{0}
\newCharapter{chapters/5projekt}
% \setcounter{figure}{0}
% \newCharapter{chapters/chapter6}
% \setcounter{figure}{0}
% \newCharapter{chapters/chapter7}
% \setcounter{figure}{0}
% \newCharapter{chapters/chapter8}

\nocite{*}
%\printbibliography[type=article,title={Źródła naukowe}]
%\printbibliography[type=online,title={Źródła internetowe}]
%\printbibliography[sorting=nyt, title={Bibliografia}]

\end{document}
