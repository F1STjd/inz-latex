\section{Analiza rynku}

Na rynku istnieje wiele witryn internetowych, na których użytkownicy mogą licytować drogocenne przedmioty. Jak okazuje się, w największych witrynach, nie ma możliwości podglądu obiektu w formie 3D. Witryny takie jak invaluable.com, desa.pl i auctionet.com posiadają prostą funkcjonalność podglądu zdjęć, nie wyróżniając się innowacyjnym rozwiązaniem jakim jest rendering 3D.

Obrazy 2D zapewniają statyczny widok i nie oddają skutecznie skomplikowanych szczegółów, kątów lub faktur. Prowadzi to do tego, że nie mamy rzeczywistego oddania stanu przedmiotu, przez co potencjalny konsument, może zostać wprowadzony w błąd, podczas zakupu danego przedmiotu. Technologia ta pozwala nam na dokładną inspekcję wybranego przez nas przedmiotu, poprzez obracanie obiektem, możliwością przybliżania oraz oddalania się od obiektu i renderingu w bardzo wysokiej jakości.

\subsection{invaluable.com}
Jest to jeden z większych serwisów internetowych, zajmujących się sprzedażą obrazów, antyków i rzeczy kolekcjonerskich. Współpracują z ponad pięcioma tysiącami domów aukcyjnych na całym świecie. Ten serwis, będąc jednym z większych na świecie, nie zapewnia renderowania 3D obiektów. Przeglądając aukcje, użytkownik może przejrzeć i przybliżyć tylko pierwsze zdjęcie, aby dokonać dokładniejszej inspekcji przedniej części obiektu, który potencjalnie będzie licytowany.

\image{img/schematy/invaluable.png}{Podgląd obiektu w serwisie invaluable.com}

\subsection{desa.pl}
Największy dom aukcyjny w Polsce, który prowadzi aukcje on-line i stacjonarnie. Głównie zajmuje się wyceną i sprzedażą dzieł sztuki. Rocznie realizują 200 aukcji, na który sprzedano 7441 dzieł sztuki. Interfejs tej strony jest prosty, intuicyjny w obsłudze dla użytkownika. Jednak podgląd obiektu licytacji jest niezaawansowany. Możemy jedynie podejrzeć zdjęcia dołączone do aukcji, lecz nie możemy ich przybliżyć i dokonać dokładniejszej inspekcji danego dzieła.

\image{img/schematy/desa.png}{Podgląd obiektu w serwisie desa.pl}

\subsection{auctionet.com}
Serwis założony w 2011 przez Niklasa Söderholma, założyciela i prezesa zarządu Bukowis Market i Tom Österman, eksperta z Bukowskis and Åmells. Obsługuje domy aukcyjne na terenie całej Europy, będąc pośrednikiem w sprzedaży dzieł sztuki i przedmiotów kolekcjonerskich. Strona z zawartością aukcji jest bardzo przejrzysta. Sam podgląd obiektu to są tylko i wyłącznie zdjęcia, które możemy przybliżać i zobaczyć kilka na raz. Przez takie rozwiązanie, inspekcja przedmiotu jest zależna od formatu i jakości zdjęcia, które mogą być w gorszej jakości, jeżeli były poddane jakiejś formie kompresji.

\img{img/schematy/auctionet.png}{Podgląd obiektu w serwisie auctionet.com}{H}