\section{Użyte technologie i narzędzia}

%–––––––––––––––––––––––––––––––––––––––––––––––––––––––––––––––––––––
\subsection{Element renderujący}

  \subsubsection{Język C++}
  Wysokopoziomowy język programowania stworzony przez Bjarne Stroustrupa, jako dodatek do języka C. C++ jest szeroko używany w aplikacjach wymagających wysokiej wydajności, takich jak gry komputerowe, silniki gier komputerowych, aplikacje graficzne, aplikacje uczenia maszynowego i wiele innych, gdzie wydajność jest kluczowa. Perfekcyjnie wpasowuję się on w potrzebę stworzenia graficznego komponentu renderującego obiekty 3D.

  W przedstawionym projekcie wykorzystywany jest C++ 23 (ISO/IEC 14882:2024), czyli aktualny otwarty standard języka. Ostateczna wersja dokumentu to N4950. Zawiera ona w sobie wiele nowych funkcji, oraz całą zawartość wcześniejszych standardów. 

  \subsubsection{WebGPU}
  Nowoczesny interfejs programowania aplikacji (API), który umożliwia wydajny dostęp do procesora graficznego (GPU) na różnych platformach. U podstawy działania wykorzystuje systemowe interfejsy: Vulkan, Metal, lub Direct3D 12. Zastępuje starszą technologię WebGL, jako nowy główny standard graficzny dla sieci.

  \begin{komentarz}
  \texttt{WebGPU} jest wspierane zarówno w Google Chrome, jak i Firefox. Mozilla używa własnej implementacji w języku Rust o nazwie \texttt{wgpu}. Google natomiast stworzyło \texttt{Dawn}, czyli implementację standardu \texttt{WebGPU} w Chromium za pomocą C++. 
  \end{komentarz}
  \todo[inline]{Komentarz: 
  1. Czy należy zaznaczać nazwy bibliotek, czcionką monospace, tak jak powyżej? Jeżeli tak to czy tylko tych podrzędnych, skoro WebGPU jest główną technologią to zostawić Times New Roman?
  2. Czy taka długość opisu wystarczy? Aktualnie jest krótki opis profilowany, aby nawiązywać do projektu.}
 
  \subsubsection{WebGPU‑Cpp} 
  Wrapper dla WebGPU napisany w C++, ponieważ obie implementacje udostępniają plik nagłówkowy w języku C z definicjami funkcji i struktur. Stworzony został przez użytkownika eliemichel i udostępniany jest na platformie Github.

  \subsubsection{WebAssembly}
  Otwarty standard, który definiuje przenośny format binarny i odpowiadający mu format tekstowy dla programów komputerowych. Głównym celem jest umożliwienie łatwiejszego tworzenia wielce wydajnych apliakcji w przeglądarkach na stronach internetowych. Jest niezależny od platformy oraz wspiera każdy język programowania. Kod wykonywany jest w wirtualnej maszynie stosowej.
  
  \subsubsection{Emscripten}
  Całkowicie otwarty kompilator, który umożliwia kompilację kodu C i C++, lub innego języka używającego LLVM, do formatu WebAssembly. Emcc, czyli frontend kompilatora używa Clang, oraz LLVM. Pozwala on na bezproblemową konwersję praktycznie każdego projektu C i C++ do WebAssembly. Emscripten ma na koncie kilka udanych konwersji takich jak: Unreal Engine 4, Quake 3, czy Doom 3, wszystkie działające w przeglądarce.

  \subsubsection{WGSL (WebGPU Shading Language)}
  \begin{komentarz}
  Wysokopoziomowy język shaderów dla WebGPU. Umożliwia on tworzenie shaderów, czyli programów wykonywanych na procesorze graficznym. Został stworzony przez W3C GPU for the Web Community Group, aby zapewnić nowoczesny, bezpieczny i przenośny sposób pisania shaderów dla WebGPU.
  \end{komentarz}
  \todo[inline]{Komentarz: 1. Czy tłumaczenie shading language jest poprawne? Nigdzie nie znalazłem innego, a język cieniowania brzmi groteskowo. 2. Ciężko jest więcej napisać bez potrzeby wyjaśniania kolejnych sformułowań, jak: możliwość kompilacji do SPIR-V}

  \subsubsection{CMake}
  Narzędzie do zarządzania budowaniem kodu źródłowego. Zdejmuje z użytkownika konieczkość pisania skomplikowanych plików potrzebnych systemom budowania. Jest szeroko używany w projektach C i C++.

  \subsubsection{Ninja}
  Lekki system budowania, skupiający się na szybkości. Jest zaprojektowany, aby pliki wejściowe były generowanie przez inne wysokopoziomowe narzędzie. Używany do budowania Google Chrome, czy części systemu Android.

  \subsubsection{Biblioteki C++}
    \paragraph{GLFW} Napisana w C wieloplatformowa biblioteka do tworzenia okien aplikacji używających OpenGL, OpenGL ES i Vulkan.
    \paragraph{GLM (OpenGL Mathematics)} Napisana w C++ biblioteka matematyczna przeznaczona dla oprogramowania graficznego opartego na specyfikacjach języka OpenGL Shading Language (GLSL). Biblioteka ta doskonale współpracuje z OpenGL, ale zapewnia kompatybilność z innymi bibliotekami i zestawami SDK innych producentów. 
    \paragraph{stb\_image} Biblioteka dla C/C++ umożliwiająca załadowywanie obrazów w różnych formatach, takich jak: JPG, PNG, TGA, BMP, PSD, GIF, HDR, PIC.
    \paragraph{tinyobjloader} Napisana w C++ biblioteka umożliwiająca załadowywanie modeli 3D w formacie OBJ stworzonym przez Wavefront Technologies.
    \paragraph{Dear ImGui} Napisana w C++ biblioteka umożliwiająca tworzenie lekkich interfejsów graficznych z minimalnymi zależnościami.

%–––––––––––––––––––––––––––––––––––––––––––––––––––––––––––––––––––––
\subsection{System aukcyjny}

  \tcbox{\dots}
  \todo[inline]{Komentarz: Uzupełnione po implementacji}
