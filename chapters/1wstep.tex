\section{Wstęp}

Z roku na rok rynek e-commerce dynamicznie się rozwija, przyciągając coraz większą liczbę użytkowników. Według badania „E-commerce w Polsce 2024” szacuje się, że w 2024 roku aż 75\% internautów dokonuje zakupów w polskich sklepach internetowych, a 36\% korzysta z zagranicznych platform e-commerce. Pandemia Covid-19 znacząco przyspieszyła ten trend --- w 2020 roku wartość sprzedaży online w Polsce podwoiła się w porównaniu z rokiem poprzednim. Wraz z rozwojem rynku użytkownicy oczekują coraz bardziej innowacyjnych rozwiązań, które zwiększają ich zaufanie do zakupów online i poprawiają jakość doświadczeń zakupowych.

Jednym z kluczowych wyzwań w handlu internetowym, w szczególności w sektorze aukcji online, jest zapewnienie kupującym możliwości dokładnego zapoznania się ze stanem oferowanego przedmiotu. Tradycyjne platformy aukcyjne, \tcbox{takie jak desa.pl, the-saleroom.com czy allegro.pl,} \todo[inline]{Czy potrzeba tutaj wymieniać nazwy, jeżeli zostaną one dokładnie opisane w analizie rynku?} opierają się głównie na zdjęciach i opisach, które nie zawsze w pełni oddają rzeczywisty stan produktu. Brak możliwości szczegółowego obejrzenia przedmiotu może prowadzić do nieporozumień i obniżać zaufanie kupujących. Rozwiązaniem tego problemu może być zastosowanie nowoczesnych technologii wizualizacji, które umożliwiają interaktywne przedstawienie produktów w trójwymiarowej przestrzeni.

W naszej pracy inżynierskiej skupiamy się na wykorzystaniu nowatorskiej technologii WebGPU do renderowania interaktywnych modeli 3D przedmiotów aukcyjnych w przeglądarce internetowej. WebGPU, będące następcą WebGL, oferuje znacznie wyższą wydajność i lepsze możliwości graficzne, umożliwiając tworzenie zaawansowanych wizualizacji przy minimalnym obciążeniu dla użytkownika. W porównaniu do starszych technologii, takich jak WebGL, które borykały się z ograniczeniami w wydajności renderowania złożonych modeli, czy nieaktualnego Adobe Flash Player, który ze względu na problemy z bezpieczeństwem i brak wsparcia przestał być stosowany, WebGPU stanowi przełom w dostarczaniu płynnych i realistycznych wrażeń wizualnych w środowisku przeglądarkowym. Dzięki temu użytkownicy mogą dokładnie obejrzeć przedmiot z każdej perspektywy, co zwiększa transparentność procesu aukcyjnego i minimalizuje ryzyko nieporozumień dotyczących stanu produktu.

\begin{komentarz}
Nasz projekt inżynierski zakłada opracowanie systemu domu aukcyjnego, w którym kluczową funkcjonalnością jest interaktywne renderowanie modeli 3D za pomocą WebGPU. System aukcyjny pełni funkcję demonstracyjną, stanowiąc praktyczny przypadek użycia dla tej technologii. Wdrożenie WebGPU pozwala na stworzenie nowoczesnego, intuicyjnego i angażującego interfejsu, który wyróżnia się na tle istniejących rozwiązań aukcyjnych, oferując użytkownikom nową jakość w doświadczeniu zakupowym.
\end{komentarz}
\todo[inline]{Czy nie to samo co w celu i zakresie pracy?}

\newpage
